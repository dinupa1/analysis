 \documentclass[10pt, xcolor={dvipsnames}, aspectratio = 169]{beamer}
%\usetheme{Berlin}
\usefonttheme{serif}
\usepackage{graphicx}
\usepackage{amsmath}
\usepackage{hyperref}
\usepackage[absolute,overlay]{textpos}
\usepackage{mathrsfs}
%\usepackage{tikz}
%\usetikzlibrary{shapes.geometric, arrows}
\usepackage[font=tiny]{caption}
%\usepackage{mathrsfs}
\usepackage[style=science]{biblatex}
%\usepackage{standalone}

%\bibliography{ref}

\mode<presentation>
{
	\setbeamercolor{frametitle}{fg=White, bg=MidnightBlue}
	\setbeamercolor{title}{fg=White, bg=MidnightBlue}
	\setbeamercolor{background canvas}{bg=White}
	\setbeamercolor{section number projected}{bg=White, fg=MidnightBlue}
	\setbeamercolor{subsection number projected}{bg=White, fg=MidnightBlue}
	\setbeamertemplate{items}{\color{MidnightBlue}$\blacksquare$}
	\setbeamercolor{text}{fg=MidnightBlue}
	\setbeamertemplate{footline}[frame number]
	\setbeamertemplate{caption}[numbered]
}

\title{2D Fit with Prop. Tube}
\author{Dinupa}

\begin{document}


\begin{frame}
\maketitle
\end{frame}

%path=/Users/dinupa/Documents/e1039/kefficiency/dummy/Efficiency/pics

%% slide 1
\begin{frame}[fragile]

\begin{textblock}{7.0}(0.5, 0.5)
\begin{figure}
\centering
\includegraphics[width=7.0cm]{../pics/nhits.png}
%\caption{Number of hits in the reco. track.}
\end{figure}
\end{textblock}

\begin{textblock}{7.0}(8.0, 0.5)
\begin{figure}
\centering
\includegraphics[width=7.0cm]{../pics/chisq.png}
%\caption{$\chi^{2}$ of hits in the reco. track.}
\end{figure}
\end{textblock}

\begin{textblock}{10.0}(0.5, 13.)
\begin{itemize}

	\item We only consider the back partial tracks.
	%\item \verb|NIM4| and \verb|FPGA5| trigger was considered.
	\item \verb|FPGA5| trigger was considered.
\end{itemize}
\end{textblock}

\end{frame}


\begin{frame}[fragile]

\begin{textblock}{7.0}(0.5, 0.5)
\begin{figure}
\centering
\includegraphics[width=7.0cm]{../pics/ele24.png}
%\caption{Number of hits in the reco. track.}
\end{figure}
\end{textblock}

\begin{textblock}{7.0}(8.0, 1.0)
\begin{itemize}
\item For many events which has a hit in the \verb|H4T| plane, does not have a hit in the \verb|H2B| plane.
\item Events with hits in both planes was considered in the 2D histogram.
\end{itemize}
\end{textblock}

\end{frame}


%% slide 2
\begin{frame}

\begin{figure}
\centering
\includegraphics[width=15.0cm]{../pics/effi_45.png}
\caption{H4B}
\end{figure}

\end{frame}


\begin{frame}

\begin{figure}
\centering
\includegraphics[width=15.0cm]{../pics/effi1_45.png}
\caption{H4B}
\end{figure}

\end{frame}


\begin{frame}

\begin{figure}
\centering
\includegraphics[width=15.0cm]{../pics/effi2_45.png}
\caption{H4B}
\end{figure}

\end{frame}

%% slide 5
\begin{frame}

\begin{figure}
\centering
\includegraphics[width=15.0cm]{../pics/effi_46.png}
\caption{H4T}
\end{figure}

\end{frame}


\begin{frame}

\begin{figure}
\centering
\includegraphics[width=15.0cm]{../pics/effi1_46.png}
\caption{H4T}
\end{figure}

\end{frame}


\begin{frame}

\begin{figure}
\centering
\includegraphics[width=15.0cm]{../pics/effi2_46.png}
\caption{H4T}
\end{figure}

\end{frame}

%% slide 6
\begin{frame}

\begin{figure}
\centering
\includegraphics[width=15.0cm]{../pics/effi_41.png}
\caption{H4Y1L}
\end{figure}

\end{frame}


\begin{frame}

\begin{figure}
\centering
\includegraphics[width=15.0cm]{../pics/effi1_41.png}
\caption{H4Y1L}
\end{figure}

\end{frame}


\begin{frame}

\begin{figure}
\centering
\includegraphics[width=15.0cm]{../pics/effi2_41.png}
\caption{H4Y1L}
\end{figure}

\end{frame}

%% slide 7
\begin{frame}

\begin{figure}
\centering
\includegraphics[width=15.0cm]{../pics/effi_42.png}
\caption{H4Y1R}
\end{figure}

\end{frame}

\begin{frame}

\begin{figure}
\centering
\includegraphics[width=15.0cm]{../pics/effi1_42.png}
\caption{H4Y1R}
\end{figure}

\end{frame}

\begin{frame}

\begin{figure}
\centering
\includegraphics[width=15.0cm]{../pics/effi2_42.png}
\caption{H4Y1R}
\end{figure}

\end{frame}

%% slide 8
\begin{frame}

\begin{figure}
\centering
\includegraphics[width=15.0cm]{../pics/effi_43.png}
\caption{H4Y2L}
\end{figure}

\end{frame}

\begin{frame}

\begin{figure}
\centering
\includegraphics[width=15.0cm]{../pics/effi1_43.png}
\caption{H4Y2L}
\end{figure}

\end{frame}

\begin{frame}

\begin{figure}
\centering
\includegraphics[width=15.0cm]{../pics/effi2_43.png}
\caption{H4Y2L}
\end{figure}

\end{frame}

%% slide 9
\begin{frame}

\begin{figure}
\centering
\includegraphics[width=15.0cm]{../pics/effi_44.png}
\caption{H4Y2R}
\end{figure}

\end{frame}

\begin{frame}

\begin{figure}
\centering
\includegraphics[width=15.0cm]{../pics/effi1_44.png}
\caption{H4Y2R}
\end{figure}

\end{frame}

\begin{frame}

\begin{figure}
\centering
\includegraphics[width=15.0cm]{../pics/effi2_44.png}
\caption{H4Y2R}
\end{figure}

\end{frame}
\end{document}
